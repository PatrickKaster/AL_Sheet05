% HMC Math dept HW template example
% v0.04 by Eric J. Malm, 10 Mar 2005
\documentclass[10pt,a4paper,boxed]{hmcpset}

% set 1-inch margins in the document
% \usepackage[margin=1in]{geometry}
\usepackage{enumerate}
\usepackage{todonotes}
%\usepackage{tikz}
%\usetikzlibrary{positioning}
\usepackage{subfig} % subfigures in figures.	
\usepackage{pgfplots}
\usepackage{amsmath}
\usepackage{amsfonts}
\usepackage{amssymb}

%% work around for subfig and asy environment
\makeatletter
\newsavebox{\sfe@box}
\newenvironment{subfloatenv}[2]{%
\def\sfe@caption{#1}%
\def\sfe@label{#2}%
\setbox\sfe@box\hbox\bgroup\color@setgroup}%
{\color@endgroup\egroup\subfloat[\sfe@caption]%
{\usebox{\sfe@box}\label{\sfe@label}}}
\makeatother

% include this if you want to import graphics files with /includegraphics
\usepackage{graphicx}

\renewcommand*{\familydefault}{\sfdefault}
\newcommand{\vect}[1]{\mathbf{#1}}


%\tikzset{node distance=2cm, inner/.style={draw,circle}, leaf/.style={draw,rectangle}}

\usepackage{hyperref}

% info for header block in upper right hand corner
\name{Lukas Gesing, Patrick Kaster}
\class{MA-INF 4201 - Artificial Life}
\assignment{Exercise Sheet 5}
% \duedate{09/03/2004}

\begin{document}

\begin{problem}[Assignment 31]
\end{problem}
\begin{solution}
Let $P_s$ be the population at start, and $P_e$ be the end population.
From the lecture we derive the exponential growth after one year as: $P_e = P_s \left( 1 + \frac{1.7}{100} \right)$.
After $n$ years we have a population of: $P_e = P_s \left( 1 + \frac{1.7}{100} \right)^n$. Since we seek $P_e = 2 P_s$, we obtain
\begin{align*}
	2 P_s & = P_s \left( 1 + \frac{1.7}{100} \right)^n & \Leftrightarrow \\
	2 & = \left( 1 + \frac{1.7}{100} \right)^n & \Leftrightarrow \\
	\log(2) & = \log \left( 1 + \frac{1.7}{100} \right) n & \Leftrightarrow \\
	n & = \frac{\log(2)}{\log \left( 1 + \frac{1.7}{100} \right)} & \Rightarrow \\
	n & \eqsim 41.12 (a)
\end{align*}
\end{solution}

\begin{problem}[Assignment 32]
\end{problem}
\begin{solution}
\end{solution}

\begin{problem}[Assignment 33]
\end{problem}
\begin{solution}
\end{solution}

\begin{problem}[Assignment 34]
\end{problem}
\begin{solution}
Referring to our last exercise sheet, we constructed an L-system for a spiral basically by adding one unit length to a segment compromising a quarter turn of the spiral. A golden spiral is a spiral where in each quarter turn the golden ratio is added, in order for the spiral to unwind. In other words: Inscribing each quarter turn segment into a bounding square, the area ratio of squares of consecutive quarter turns is the golden ratio.
\end{solution}

\begin{problem}[Assignment 35]
\end{problem}
\begin{solution}
Suppose the limit exists and call it $a$: $\lim_{n \rightarrow \infty} \frac{F_{n+1}}{F_n} = a$. Then:
\begin{align*}
	a = \lim_{n \rightarrow \infty} \frac{F_{n+1}}{F_n} & = \lim_{n \rightarrow \infty} \frac{F_{n}+F_{n-1}}{F_n} = 1 + \lim_{n \rightarrow \infty} \frac{F_{n-1}}{F_n} = 1 + \frac{1}{a} \\
	& \Rightarrow 0 = a^2 - a - 1 
\end{align*}
Using the midnight equation we obtain the solutions: $\frac{1 \pm \sqrt{5}}{2}$.

The program implemented in Python is found in the file \verb|fib.py| and takes the number of iterations as parameter \verb|-n|.
\end{solution}

\begin{problem}[Assignment 36]
\end{problem}
\begin{solution}
\end{solution}

\end{document}
